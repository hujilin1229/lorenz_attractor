% Generated by Sphinx.
\def\sphinxdocclass{report}
\documentclass[letterpaper,10pt,english]{sphinxmanual}

\usepackage[utf8]{inputenc}
\ifdefined\DeclareUnicodeCharacter
  \DeclareUnicodeCharacter{00A0}{\nobreakspace}
\else\fi
\usepackage{cmap}
\usepackage[T1]{fontenc}
\usepackage{amsmath,amssymb}
\usepackage{babel}
\usepackage{times}
\usepackage[Bjarne]{fncychap}
\usepackage{longtable}
\usepackage{sphinx}
\usepackage{multirow}
\usepackage{eqparbox}


\addto\captionsenglish{\renewcommand{\figurename}{Fig. }}
\addto\captionsenglish{\renewcommand{\tablename}{Table }}
\SetupFloatingEnvironment{literal-block}{name=Listing }

\addto\extrasenglish{\def\pageautorefname{page}}

\setcounter{tocdepth}{1}


\title{Mini-Project for Phd-Python-1 Documentation}
\date{Jul 28, 2016}
\release{0.0.1}
\author{Jilin Hu}
\newcommand{\sphinxlogo}{}
\renewcommand{\releasename}{Release}
\makeindex


\begin{document}

\maketitle
\tableofcontents
\phantomsection\label{index::doc}


\chapter{Problem Statement}
\label{ProblemStatement:problem-statement}\label{ProblemStatement:welcome-to-mini-project-for-phd-python-1-s-documentation}\label{ProblemStatement::doc}
This mini-project relates to the Lorenz attractor. In this process, we need to solve the Ordinary Differential Equations
(ODEs), write and read the data, and plot the data. By looking this problem, we can divide this problem into following steps.


\section{ODE Solver}
\label{ProblemStatement:ode-solver}
In this part, we construct a class named ``ODE\_solvers''. In this class, there's a initial function which is used to
specify sigma, beta, rho of the ODE and these three parameters are the attributes of this class. Moreover, we also
define a function named ``Euler\_solve'' which adopts Euler approach to solve this ODE by giving the initial conditions
corresponding to values for (x{[}0{]}, y{[}0{]}, z{[}0{]}), time interval and number of steps. Of course, you can added some more
advanced and accurate solve methods into this class, but here we just implement the Euler one to show how it works.


\section{Filehandling}
\label{ProblemStatement:filehandling}
In this part, we implement a module named ``filehandling'' to take over the task of writing(reading)the trajectory from
``Euler\_solve'' into(out) file.


\section{Plotting}
\label{ProblemStatement:plotting}
In this part, we implement a module named ``plot'' which can plot a 3D and 2D of the trajectory

\section{Running}
\label{ProblemStatement:running}
This is a integrated module, which integrate the solver, file writing and reading, and plotting. There are two methods
in this module, ``run\_save\_plot'' and ``run\_load\_plot''. When you call ``run\_save\_plot'', it will automatically solving the
ODE, saving the data into file and plot the trajectory. Otherwise, you can call ``run\_load\_plot'', it will load the data
from the file you specified and plot the trajectory.

\section{Testing}
\label{ProblemStatement:testing}
Here we conduct unit test for the functions mentioned above. All of these functions passed the unit test.

\chapter{lorenz}
\label{modules:lorenz}\label{modules::doc}
\begin{longtable}{ll}
\hline
\endfirsthead

\multicolumn{2}{c}%
{{\tablecontinued{\tablename\ \thetable{} -- continued from previous page}}} \\
\hline
\endhead

\hline \multicolumn{2}{|r|}{{\tablecontinued{Continued on next page}}} \\ \hline
\endfoot

\endlastfoot


{\hyperref[_autosummary/lorenz:module\string-lorenz]{\crossref{\code{lorenz}}}}
 & 

\\
\hline\end{longtable}



\section{lorenz package}
\label{_autosummary/lorenz::doc}\label{_autosummary/lorenz:lorenz-package}

\subsection{Submodules}
\label{_autosummary/lorenz:submodules}

\subsection{lorenz.filehandling module}
\label{_autosummary/lorenz:module-lorenz.filehandling}\label{_autosummary/lorenz:lorenz-filehandling-module}\index{lorenz.filehandling (module)}
This file can contain functionalities for saving/loading data
\paragraph{Functions}

\begin{longtable}{ll}
\hline
\endfirsthead

\multicolumn{2}{c}%
{{\tablecontinued{\tablename\ \thetable{} -- continued from previous page}}} \\
\hline
\endhead

\hline \multicolumn{2}{|r|}{{\tablecontinued{Continued on next page}}} \\ \hline
\endfoot

\endlastfoot


{\hyperref[_autosummary/lorenz:lorenz.filehandling.write_x_y_z_tofile]{\crossref{\code{write\_x\_y\_z\_tofile}}}}(var, filename)
 & 
Write the arrays of x, y, z to a file
\\
\hline
{\hyperref[_autosummary/lorenz:lorenz.filehandling.read_x_y_z_fromfile]{\crossref{\code{read\_x\_y\_z\_fromfile}}}}(filename)
 & 
read x y z from file: filename
\\
\hline\end{longtable}

\index{write\_x\_y\_z\_tofile() (in module lorenz.filehandling)}

\begin{fulllineitems}
\phantomsection\label{_autosummary/lorenz:lorenz.filehandling.write_x_y_z_tofile}\pysiglinewithargsret{\code{lorenz.filehandling.}\bfcode{write\_x\_y\_z\_tofile}}{\emph{var}, \emph{filename}}{}
Write the arrays of x, y, z to a file
\begin{quote}\begin{description}
\item[{Parameters}] \leavevmode\begin{itemize}
\item {} 
\textbf{\texttt{var}} -- x, y, z

\item {} 
\textbf{\texttt{filename}} -- the target filename

\end{itemize}

\item[{Returns}] \leavevmode
None

\item[{Return type}] \leavevmode
\href{https://docs.python.org/library/constants.html\#None}{None}

\end{description}\end{quote}

\end{fulllineitems}

\index{read\_x\_y\_z\_fromfile() (in module lorenz.filehandling)}

\begin{fulllineitems}
\phantomsection\label{_autosummary/lorenz:lorenz.filehandling.read_x_y_z_fromfile}\pysiglinewithargsret{\code{lorenz.filehandling.}\bfcode{read\_x\_y\_z\_fromfile}}{\emph{filename}}{}
read x y z from file: filename
\begin{quote}\begin{description}
\item[{Parameters}] \leavevmode
\textbf{\texttt{filename}} -- The file which we want to read from

\item[{Returns}] \leavevmode
three arrays: x, y, z

\item[{Return type}] \leavevmode
numpy.arrays

\end{description}\end{quote}

\end{fulllineitems}



\subsection{lorenz.plot module}
\label{_autosummary/lorenz:lorenz-plot-module}\label{_autosummary/lorenz:module-lorenz.plot}\index{lorenz.plot (module)}
This file may contain functionalities for plotting
\paragraph{Functions}

\begin{longtable}{ll}
\hline
\endfirsthead

\multicolumn{2}{c}%
{{\tablecontinued{\tablename\ \thetable{} -- continued from previous page}}} \\
\hline
\endhead

\hline \multicolumn{2}{|r|}{{\tablecontinued{Continued on next page}}} \\ \hline
\endfoot

\endlastfoot


{\hyperref[_autosummary/lorenz:lorenz.plot.plot3Dpdf]{\crossref{\code{plot3Dpdf}}}}({[}x, y, z, sigma, beta, rho{]})
 & 
Plot the 3D scatter of lorenz
\\
\hline
{\hyperref[_autosummary/lorenz:lorenz.plot.plot2Dpdf]{\crossref{\code{plot2Dpdf}}}}({[}x, y, z, sigma, beta, rho{]})
 & 
Plot 3*1 subplots of 2D plot for lorenz
\\
\hline\end{longtable}

\index{plot3Dpdf() (in module lorenz.plot)}

\begin{fulllineitems}
\phantomsection\label{_autosummary/lorenz:lorenz.plot.plot3Dpdf}\pysiglinewithargsret{\code{lorenz.plot.}\bfcode{plot3Dpdf}}{\emph{x=None}, \emph{y=None}, \emph{z=None}, \emph{sigma=10}, \emph{beta=2.67}, \emph{rho=6}}{}
Plot the 3D scatter of lorenz
\begin{quote}\begin{description}
\item[{Parameters}] \leavevmode\begin{itemize}
\item {} 
\textbf{\texttt{x}} -- x-axis data

\item {} 
\textbf{\texttt{y}} -- y-axis data

\item {} 
\textbf{\texttt{z}} -- z-axis data

\item {} 
\textbf{\texttt{sigma}} -- param sigma for lorenz

\item {} 
\textbf{\texttt{beta}} -- param beta for lorenz

\item {} 
\textbf{\texttt{rho}} -- param rho for lorenz

\end{itemize}

\item[{Returns}] \leavevmode
bool

\end{description}\end{quote}

\end{fulllineitems}

\index{plot2Dpdf() (in module lorenz.plot)}

\begin{fulllineitems}
\phantomsection\label{_autosummary/lorenz:lorenz.plot.plot2Dpdf}\pysiglinewithargsret{\code{lorenz.plot.}\bfcode{plot2Dpdf}}{\emph{x=None}, \emph{y=None}, \emph{z=None}, \emph{sigma=10}, \emph{beta=2.67}, \emph{rho=6}}{}
Plot 3*1 subplots of 2D plot for lorenz
\begin{quote}\begin{description}
\item[{Parameters}] \leavevmode\begin{itemize}
\item {} 
\textbf{\texttt{x}} -- x-axis data

\item {} 
\textbf{\texttt{y}} -- y-axis data

\item {} 
\textbf{\texttt{z}} -- z-axis data

\item {} 
\textbf{\texttt{sigma}} -- param sigma for lorenz

\item {} 
\textbf{\texttt{beta}} -- param beta for lorenz

\item {} 
\textbf{\texttt{rho}} -- param rho for lorenz

\end{itemize}

\item[{Returns}] \leavevmode
bool

\end{description}\end{quote}

\end{fulllineitems}



\subsection{lorenz.run module}
\label{_autosummary/lorenz:lorenz-run-module}\label{_autosummary/lorenz:module-lorenz.run}\index{lorenz.run (module)}
This file may contain a convenient interface/function for

1: computing a trajectory using an ODE solver from solver.py
2: save data to file
3: plot data

and possible another function that

2: load data from file
3: plot data
\paragraph{Functions}

\begin{longtable}{ll}
\hline
\endfirsthead

\multicolumn{2}{c}%
{{\tablecontinued{\tablename\ \thetable{} -- continued from previous page}}} \\
\hline
\endhead

\hline \multicolumn{2}{|r|}{{\tablecontinued{Continued on next page}}} \\ \hline
\endfoot

\endlastfoot


{\hyperref[_autosummary/lorenz:lorenz.run.run_save_plot]{\crossref{\code{run\_save\_plot}}}}(sigma, beta, rho, initial\_arrays)
 & 
Run Euler\_solve, save and plot trajectory
\\
\hline
{\hyperref[_autosummary/lorenz:lorenz.run.run_load_plot]{\crossref{\code{run\_load\_plot}}}}(target\_file, sigma, beta, rho)
 & 
Read and plot trajectory
\\
\hline\end{longtable}

\index{run\_save\_plot() (in module lorenz.run)}

\begin{fulllineitems}
\phantomsection\label{_autosummary/lorenz:lorenz.run.run_save_plot}\pysiglinewithargsret{\code{lorenz.run.}\bfcode{run\_save\_plot}}{\emph{sigma}, \emph{beta}, \emph{rho}, \emph{initial\_arrays}}{}
Run Euler\_solve, save and plot trajectory
\begin{quote}\begin{description}
\item[{Parameters}] \leavevmode\begin{itemize}
\item {} 
\textbf{\texttt{sigma}} -- sigma

\item {} 
\textbf{\texttt{beta}} -- beta

\item {} 
\textbf{\texttt{rho}} -- rho

\item {} 
\textbf{\texttt{initial\_arrays}} -- {[}x{[}0{]}, y{[}0{]}, z{[}0{]}{]}

\end{itemize}

\item[{Returns}] \leavevmode
the name of saved file.

\item[{Return type}] \leavevmode
String

\end{description}\end{quote}

\end{fulllineitems}

\index{run\_load\_plot() (in module lorenz.run)}

\begin{fulllineitems}
\phantomsection\label{_autosummary/lorenz:lorenz.run.run_load_plot}\pysiglinewithargsret{\code{lorenz.run.}\bfcode{run\_load\_plot}}{\emph{target\_file}, \emph{sigma}, \emph{beta}, \emph{rho}}{}
Read and plot trajectory
\begin{quote}\begin{description}
\item[{Parameters}] \leavevmode\begin{itemize}
\item {} 
\textbf{\texttt{target\_file}} -- The pickle file contain the traj of x, y, z

\item {} 
\textbf{\texttt{sigma}} -- the sigma used for this trajectory

\item {} 
\textbf{\texttt{beta}} -- the beta used for this trajectory

\item {} 
\textbf{\texttt{rho}} -- the rho used for this trajectory

\end{itemize}

\item[{Returns}] \leavevmode
None

\item[{Return type}] \leavevmode
\href{https://docs.python.org/library/constants.html\#None}{None}

\end{description}\end{quote}

\end{fulllineitems}



\subsection{lorenz.solver module}
\label{_autosummary/lorenz:module-lorenz.solver}\label{_autosummary/lorenz:lorenz-solver-module}\index{lorenz.solver (module)}
This file may contain the ODE solver
\paragraph{Classes}

\begin{longtable}{ll}
\hline
\endfirsthead

\multicolumn{2}{c}%
{{\tablecontinued{\tablename\ \thetable{} -- continued from previous page}}} \\
\hline
\endhead

\hline \multicolumn{2}{|r|}{{\tablecontinued{Continued on next page}}} \\ \hline
\endfoot

\endlastfoot


{\hyperref[_autosummary/lorenz:lorenz.solver.ODE_solvers]{\crossref{\code{ODE\_solvers}}}}(sigma, beta, rho)
 & 
An integrated ODE solvers
\\
\hline\end{longtable}

\index{ODE\_solvers (class in lorenz.solver)}

\begin{fulllineitems}
\phantomsection\label{_autosummary/lorenz:lorenz.solver.ODE_solvers}\pysiglinewithargsret{\strong{class }\code{lorenz.solver.}\bfcode{ODE\_solvers}}{\emph{sigma}, \emph{beta}, \emph{rho}}{}
Bases: \href{https://docs.python.org/library/functions.html\#object}{\code{object}}

An integrated ODE solvers
By giving the initial parameters, we can construct the ODE solver
\begin{description}
\item[{Example:}] \leavevmode
solver = ODE\_solver(sigma, beta, rho)

\end{description}
\index{Euler\_solve() (lorenz.solver.ODE\_solvers method)}

\begin{fulllineitems}
\phantomsection\label{_autosummary/lorenz:lorenz.solver.ODE_solvers.Euler_solve}\pysiglinewithargsret{\bfcode{Euler\_solve}}{\emph{initial\_arrays}, \emph{N=50000}, \emph{t\_sigma=0.01}}{}
Solve the equation with Euler approach
\begin{quote}\begin{description}
\item[{Parameters}] \leavevmode\begin{itemize}
\item {} 
\textbf{\texttt{initial\_arrays}} -- list of initial values for x{[}0{]}, y{[}0{]}, z{[}0{]}

\item {} 
\textbf{\texttt{N}} -- the total steps to calculate the equation

\item {} 
\textbf{\texttt{t\_sigma}} -- the minimal step

\end{itemize}

\item[{Returns}] \leavevmode
arrays of {[}x, y, z{]}, size(3*n)

\item[{Return type}] \leavevmode
numpy.arrays

\end{description}\end{quote}

\end{fulllineitems}


\end{fulllineitems}



\subsection{lorenz.util module}
\label{_autosummary/lorenz:module-lorenz.util}\label{_autosummary/lorenz:lorenz-util-module}\index{lorenz.util (module)}
This file may contain utility functionalities to the extend you will need it
\paragraph{Functions}

\begin{longtable}{ll}
\hline
\endfirsthead

\multicolumn{2}{c}%
{{\tablecontinued{\tablename\ \thetable{} -- continued from previous page}}} \\
\hline
\endhead

\hline \multicolumn{2}{|r|}{{\tablecontinued{Continued on next page}}} \\ \hline
\endfoot

\endlastfoot


{\hyperref[_autosummary/lorenz:lorenz.util.mkdir_p]{\crossref{\code{mkdir\_p}}}}(mypath)
 & 
Creates a directory.
\\
\hline\end{longtable}

\index{mkdir\_p() (in module lorenz.util)}

\begin{fulllineitems}
\phantomsection\label{_autosummary/lorenz:lorenz.util.mkdir_p}\pysiglinewithargsret{\code{lorenz.util.}\bfcode{mkdir\_p}}{\emph{mypath}}{}
Creates a directory. equivalent to using mkdir -p on the command line

\end{fulllineitems}



\subsection{Module contents}
\label{_autosummary/lorenz:module-lorenz}\label{_autosummary/lorenz:module-contents}\index{lorenz (module)}

\chapter{Indices and tables}
\label{index:indices-and-tables}\begin{itemize}
\item {} 
\DUrole{xref,std,std-ref}{genindex}

\item {} 
\DUrole{xref,std,std-ref}{modindex}

\item {} 
\DUrole{xref,std,std-ref}{search}

\end{itemize}


\renewcommand{\indexname}{Python Module Index}
\begin{theindex}
\def\bigletter#1{{\Large\sffamily#1}\nopagebreak\vspace{1mm}}
\bigletter{l}
\item {\texttt{lorenz}}, \pageref{_autosummary/lorenz:module-lorenz}
\item {\texttt{lorenz.filehandling}}, \pageref{_autosummary/lorenz:module-lorenz.filehandling}
\item {\texttt{lorenz.plot}}, \pageref{_autosummary/lorenz:module-lorenz.plot}
\item {\texttt{lorenz.run}}, \pageref{_autosummary/lorenz:module-lorenz.run}
\item {\texttt{lorenz.solver}}, \pageref{_autosummary/lorenz:module-lorenz.solver}
\item {\texttt{lorenz.util}}, \pageref{_autosummary/lorenz:module-lorenz.util}
\end{theindex}

\renewcommand{\indexname}{Index}
\printindex
\end{document}
